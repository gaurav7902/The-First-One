\documentclass{article}
\usepackage{hyperref}
\usepackage{amsmath}
\usepackage{graphicx}
\usepackage[utf8]{inputenc}


\title{An Overview of Finding Maxima and Minima of Functions}
\author{Gaurav Patidar}
\date{November 1, 2024}

\begin{document}

\maketitle

\tableofcontents
\newpage
\section{Introduction}
\label{sec:1}
Understanding how to find the \textbf{maxima} and \textbf{minima} of functions is a key concept in \textit{calculus}. It helps in identifying the highest or lowest points on a graph, which can have various applications in \underline{physics, economics, and optimization problems}.
\newpage
\section{Mathematical Formulation}
To find the maxima or minima of a function f(x), we start by finding its \underline{first derivative}:
\begin{equation}
f'(x) = \text{Derivative of } f(x).
\end{equation}

Next, we find the critical points by solving:
\begin{equation}
f'(x) = 0.
\end{equation}

The second derivative test is used to determine whether a critical point is a maximum or minimum:
\begin{itemize}
    \item If \( f''(x) > 0 \) , then it is a minimum.
    \item If \( f''(x) < 0 \), then it is a maximum.
\end{itemize}
\newpage
\section{Example: Finding the Maxima and Minima of a Quadratic Function}
Consider the function \(f(x) = -x^2 + 4x - 3\). We first calculate the derivative and find the critical points.

The \underline{first derivative}\textbf{} is:
\begin{equation}
f'(x) = -2x + 4.
\end{equation}

Solving for \(f'(x) = 0\):
\begin{equation}
-2x + 4 = 0 \Rightarrow x = 2.
\end{equation}

The second derivative is:
\begin{equation}
f''(x) = -2.
\end{equation}

Since \(f''(2) < 0\), \(x = 2\) is a maximum point.

\begin{figure}[h]
    \centering
    \includegraphics[width=0.5\textwidth]{maxima-minima.png}
    \caption{Graph showing the maxima of the function \(f(x) = -x^2 + 4*x - 3\).}
\end{figure}

\begin{table}[h]
    \centering
    \begin{tabular}{|c|c|}
        \hline
        \(x\) & \(f(x)\) \\
        \hline
        1 & 0 \\
        2 & 1 \\
        3 & 0 \\
        \hline
    \end{tabular}
    \caption{Values of \(f(x)\) at different points}
\end{table}

As we can see in Figure 1 and Table 1, the maximum value occurs at \(x = 2\).

\end{document}
